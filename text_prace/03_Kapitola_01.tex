\chapter{Relativistic quantum mechanics}

In this chapter (TODO)

\section{Canonical quantization}

\paragraph{} Canonical quantization is a process of transition from a classical theory to
the quantum one, i.e. procedure for quantizing the theory. The term \textit{canonical} comes
from (TODO).

\begin{definition}
    \label{df:poisson}
    Given two functions $f(\vec{q}, \vec{p}, t)$ and $g(\vec{q}, \vec{p}, t)$, the binary operation
    $\{f, g\}$ is called the \textbf{Poisson bracket} of these functions and takes the form

    \begin{equation}
        \{f, g\} = \sum_{i=1}^{N}\left(\partialder{f}{q_{i}} \partialder{g}{p_{i}} - \partialder{f}{p_{i}} \partialder{g}{q_{i}}\right)
    \end{equation}
\end{definition}

\begin{theorem}
    \label{th:poisson_cononical_coordinates}
    The Poisson brackets of canonical coordinates $q_{i}$ and $p_{i}$ are

    \begin{equation}
        \begin{gathered}
        \{q_{i}, q_{j}\} = 0 \\
        \{p_{i}, p_{j}\} = 0 \\
        \{q_{i}, p_{j}\} = \delta_{ij}
        \end{gathered}
    \end{equation}
\end{theorem}

\begin{proof}
    The proof of \ref{th:poisson_cononical_coordinates} can be easily done from the definition \ref{df:poisson}. 
    The first two equations are manifestly zero because $\partialder{q_{i}}{p_{j}} = \partialder{p_{i}}{q_{j}} = 0$
    and the last can one be proved as follows.

    \begin{equation}
        \begin{split}
            \{q_{i}, p_{j}\} = \sum_{k=1}^{N}\left(\partialder{q_{i}}{q_{k}} \partialder{p_{j}}{p_{k}} - \partialder{q_{i}}{p_{k}} \partialder{p_{j}}{q_{k}}\right) 
            = \sum_{k=1}^{N} \partialder{q_{i}}{q_{k}} \partialder{p_{j}}{p_{k}} = 
            \begin{cases}
                1  & \quad \text{if } i = j = k \\
                0  & \quad \text{otherwise}
            \end{cases} \\
            \implies \{q_{i}, p_{j}\} = \delta_{ij}
        \end{split}
    \end{equation}
\end{proof}

\paragraph{} Dirac has formulated a technique for generating quantum operators from
classical functions. The famous statement goes as follows.

\begin{equation}
    \{A, B\} \to \frac{1}{i \hbar}[\hat{A}, \hat{B}]
\end{equation}

For the next discussion let's define $\mathcal{O}$ map which takes a function from the configuration space
and maps it to the Hilbert space.

\begin{definition}
    \label{df:o_map}

    Let $\mathcal{Q}$ be a configuration space and $H$ a Hilbert space. $End(V)$ is an \textit{endomorphism} of space $V$, i.e. 
    a set of linear operators $\{L: V \to V \ | \  L \ linear\}$.

    \begin{equation}
        \begin{gathered}
            \mathcal{O}: Q \to End(H) \\
            \mathcal{O}(\{f, g\}) = \frac{1}{i \hbar}[\hat{A}, \hat{B}]
        \end{gathered}
    \end{equation}

    Where $\hat{A}$ is the operator corresponding the the classical value $f$ and $B$ corresponds to $g$.
\end{definition}

Now we will use the fact that $[\hat{q}_{i}, \hat{p}_{j}] = i \hbar \delta_{ij}$ to show that $\mathcal{O}(q_{i}) = \hat{q}_{i}$. Firstly, we will
rewrite $q$ using the Poisson bracket. The obvious choice is $q_{i} = \{\frac{q_{i}^{2}}{2}, p_{i}\}$.  
And from the definition \ref{df:o_map} we see that $\mathcal{O}(\{\frac{q_{i}^{2}}{2}, p_{i}\}) = \frac{1}{i \hbar}[\frac{\hat{q}_{i}^{2}}{2}, \hat{p}_{i}]$.
Using the identity for commutation relation $[AB, C] = [A, B]C + B[A, C]$\footnote{Proof by directly using the definition of 
the commutator: $[AB, C] = ABC - CAB = ABC - ACB + ACB - CAB = A [B, C] + [A, C] B$} the commutator on the right-hand side can
be written as follows.

\begin{equation*}
    \mathcal{O}(x_{i}) = \frac{1}{i \hbar}\bigg[\frac{\hat{x}_{i}^{2}}{2}, \hat{p}_{i}\bigg] = 
    \frac{1}{2 i \hbar} \big(\hat{x}_{i} [\hat{x}_{i}, \hat{p}_{i}] + [\hat{x}_{i}, \hat{p}_{i}] \hat{x}_{i}\big) =
    \frac{1}{2 i \hbar} \big(\hat{x}_{i} i \hbar + i \hbar \hat{x}_{i}\big) = \hat{x}_{i}
\end{equation*}

The proof of $\mathcal{O}(p) = \hat{p}$ is very similar.

\begin{equation*}
    \mathcal{O}(p_{i}) = \frac{1}{i \hbar}\bigg[\hat{x}_{i}, \frac{\hat{p}_{i}^{2}}{2}\bigg] = 
    \frac{1}{2 i \hbar} \big([\hat{x}_{i}, \hat{p}_{i}] \hat{p}_{i} + \hat{p}_{i} [\hat{x}_{i}, \hat{p}_{i}]\big) =
    \frac{1}{2 i \hbar} \big(i \hbar \hat{p}_{i} + \hat{p}_{i} i \hbar \big) = \hat{p}_{i}
\end{equation*}

In the \textit{x-representation} the explicit form of $\hat{x}$ and $\hat{p}$ is $\hat{x} = x$ and
$\hat{p_{x}} = - i \hbar \partialder{}{x}$.

TODO: study $\mathcal{O}(\{f(\vec{q}, \vec{p})\})$ and $\mathcal{O}(\{f + g, h\})$. How to get $\mathcal{O}(E)$?

\section{Klein-Gordon equation}

\paragraph{} \textit{Klein-Gordon equation} is relativistic quantum mechanical equation named after Oskar Klein and Walter Gordon and
it describes relativistic spinless particles. It can be found using quantization of the relativistic energy equation. We need to apply the
$\mathcal{O}$ map on each side while taking into account that $\mathcal{O}(E^2) = - \hbar^{2} \partialdernd{}{t}$ and 
$\mathcal{O}(\vec{p}^2) = - \hbar^{2} \Delta$.

\begin{equation*}
    \begin{gathered}
        E^{2} = \vec{p}^{2} c^{2} + m^{2} c^{4} \\
        - \hbar^{2} \partialdernd{}{t} \psi = - \hbar^{2} c^{2} \Delta \psi + m^{2} c^{4} \psi
    \end{gathered}
\end{equation*}

After a little rearranging and setting $\frac{m^{2} c^{2}}{\hbar^{2}} = \mu^{2}$, we will finaly get the famous form \ref{eq:klein_gordon}.

\begin{equation}
    \label{eq:klein_gordon}
    \bigg(\frac{1}{c^{2}} \partialdernd{}{t} - \Delta + \mu^{2} \bigg) \psi = 0
\end{equation}

Let's rewrite the equation in a form so we can see it is \textit{lorentz invariant}. We use the fact that \textit{4-gradient}
transforms the same way as any other 4-vector and than express the Klein-Gordon equation using 4-gradient dot products.
Using the metric $\eta^{\mu \nu} = diag(1, -1, -1, -1)$ and the \textit{4-gradient} $\partial_{\mu} = (\partialder{}{t}, \partialder{}{x}, \partialder{}{y}, \partialder{}{z})$ 
it is possible to write the term $\frac{1}{c^{2}} \partialdernd{}{t} - \Delta$ as follows.

\begin{equation*}
     \frac{1}{c^{2}} \partialdernd{}{t} - \partialdernd{}{x} - \partialdernd{}{y} - \partialdernd{}{z} = \eta^{\mu \nu} \partial_{\mu} \partial_{\nu} = \partial_{\nu} \partial^{\nu}
\end{equation*}

\begin{equation}
    \bigg(\partial_{\nu} \partial^{\nu} + \mu^{2} \bigg) \psi = 0
\end{equation}

\section{Dirac equation}

\paragraph{} One important fact about the Klein-Gordon equation is that it is second-order in time. Thus we need
one more initial condition compared to the Schrodinger Equation. Dirac tried to find an equation that is first order
in both space and time by finding the \textit{square root} of the Klein-Gordon equation. Let's start with the equation 
\ref{eq:klein_gordon}.

\begin{equation*}
    \bigg(\frac{1}{c^{2}} \partialdernd{}{t} - \Delta + \mu^{2} \bigg) \psi = 0
\end{equation*}
