\chapter{Relativistic quantum mechanics}

In this chapter we will introduce the basics of the canonical quantization. We will start by introducing the
Poisson brackets and it's basic properties. Then we will define a quantization map $\mathcal{O}$ which maps the
classical dynamical variable into the operator within a Hilbert space. Using this process we will then
formulate the Klein-Gordon equation, an equation for describing a spinless particle in the quantum mechanics.
We will see that this equation is second order both in time and space which will result some obstacles. With the
hope of solving this issue we will construct the Dirac equation which describes the $\frac{1}{2}$-spin particle.

\section{Canonical quantization}

\paragraph{} Canonical quantization is a process of transition from a classical theory to the quantum one. If a 
quantum problem has an analog in the classical mechanics we can use the process of quantization to find the formulation
in the quantum mechanics using the analogous equations in the classical one. In the classical mechanics a state of the system
is represented by coordinates in the \textit{phase space} $(\vec{q_{1}}, \dots, \vec{q_{n}}, \vec{p_{1}}, \dots, \vec{p_{n}}, t)$.
In the quantum mechanics, the analog of $q_{i}$ and $p_{i}$ are operators $\hat{q}_{i}$ and $\hat{p}_{i}$ acting in the 
\textit{Hilbert space}.

\begin{definition}
    \label{df:poisson}
    Given two dynamical variables in the phace space $\mathcal{Q}$, \newline $f(\vec{q_{1}}, \dots, \vec{q_{n}}, \vec{p_{1}}, \dots, \vec{p_{n}}, t) 
    \in C^{\infty}(\mathcal{Q}, \mathbb{R})$ and $g(\vec{q_{1}}, \dots, \vec{q_{n}}, \vec{p_{1}}, \dots, \vec{p_{n}}, t) 
    \in C^{\infty}(\mathcal{Q}, \mathbb{R})$, the binary operation $\{f, g\}$ is called the \textbf{Poisson bracket} of these functions and takes the form

    \begin{equation}
        \{f, g\} = \sum_{i=1}^{N}\left(\partialder{f}{q_{i}} \partialder{g}{p_{i}} - \partialder{f}{p_{i}} \partialder{g}{q_{i}}\right)
    \end{equation}
\end{definition}

\begin{theorem}
    \label{th:poisson_cononical_coordinates}
    The Poisson brackets of canonical coordinates $q_{i}$ and $p_{i}$ are

    \begin{equation}
        \begin{gathered}
        \{q_{i}, q_{j}\} = 0 \\
        \{p_{i}, p_{j}\} = 0 \\
        \{q_{i}, p_{j}\} = \delta_{ij}
        \end{gathered}
    \end{equation}
\end{theorem}

\begin{proof}
    The proof of \ref{th:poisson_cononical_coordinates} can be easily done from the definition \ref{df:poisson}. 
    The first two equations are manifestly zero because $\partialder{q_{i}}{p_{j}} = \partialder{p_{i}}{q_{j}} = 0$
    and the last one can be proved as follows.

    \begin{equation}
        \begin{split}
            \{q_{i}, p_{j}\} = \sum_{k=1}^{N}\left(\partialder{q_{i}}{q_{k}} \partialder{p_{j}}{p_{k}} - \partialder{q_{i}}{p_{k}} \partialder{p_{j}}{q_{k}}\right) 
            = \sum_{k=1}^{N} \partialder{q_{i}}{q_{k}} \partialder{p_{j}}{p_{k}} = 
            \begin{cases}
                1  & \quad \text{if } i = j = k \\
                0  & \quad \text{otherwise}
            \end{cases} \\
            \implies \{q_{i}, p_{j}\} = \delta_{ij}
        \end{split}
    \end{equation}
\end{proof}

\paragraph{} Paul Dirac has formulated a technique for generating operators in $H$ from dynamical variables in $C^{\infty}(\mathcal{Q}, \mathbb{R})$. 
The famous statement goes as follows.

\begin{equation}
    \label{eq:dirac_quantization}
    \{A, B\} \to \frac{1}{i \hbar}[\hat{A}, \hat{B}]
\end{equation}

That means that if we have dynamical variables $A$ and $B$ we can get their quantum conterparts $\hat{A}$ and $\hat{B}$ and the
defining property is the equation \ref{eq:dirac_quantization}. Let's define the properties of the $\mathcal{O}$ map (see \cite{floarin_jung})
and then study some consequences.

\begin{definition}
    \label{df:o_map}

    Let $\mathcal{Q}$ be a configuration space and $H$ the corresponding Hilbert space. The quantization map is

    \begin{equation}
        \mathcal{O}: C^{\infty}(\mathcal{Q}, \mathbb{R}) \to Op(H) \\
    \end{equation}
    
    having the properties:

    \begin{enumerate}
        \item $\mathcal{O}$ is linear,
        \item $\mathcal{O}(f)$ is self-adjoint, $f \in C^{\infty}(\mathcal{Q}, \mathbb{R})$,
        \item the relation between commutator and the Poisson bracket is
            \begin{equation}
                \label{eq:dirac_quantization_2}
                \mathcal{O}(\{A, B\}) = \frac{1}{i \hbar} [\hat{A}, \hat{B}]
            \end{equation}
        \item a complete set $\{f_{1}, \dots, f_{n}\}$ is mapped to a complete set $\{\mathcal{O}(f_{1}), \dots, \mathcal{O}(f_{n})\}$,
        \item an identity $Id_{\mathcal{Q}}$ is mapped to the indentity $Id_{H} = \mathcal{O}(Id_{\mathcal{Q}})$
    \end{enumerate}
\end{definition}

It is convinient to use $[\hat{q}_{i}, \hat{p}_{j}] = i \hbar \delta_{ij}$ and \ref{eq:dirac_quantization_2} to show
that $\mathcal{O}(q_{i}) = \hat{q}_{i}$. Firstly, we will rewrite $q$ using the Poisson bracket. The obvious choice is $q_{i} = \{\frac{q_{i}^{2}}{2}, p_{i}\}$.
And from the definition \ref{eq:dirac_quantization_2} we see that $\mathcal{O}(\{\frac{q_{i}^{2}}{2}, p_{i}\}) = \frac{1}{i \hbar}[\frac{\hat{q}_{i}^{2}}{2}, \hat{p}_{i}]$.
Using the identity for commutation relation $[AB, C] = [A, B]C + B[A, C]$\footnote{Proof by directly using the definition of 
the commutator: $[AB, C] = ABC - CAB = ABC - ACB + ACB - CAB = A [B, C] + [A, C] B$} the commutator on the right-hand side can
be written as follows.

\begin{equation*}
    \mathcal{O}(x_{i}) = \frac{1}{i \hbar}\bigg[\frac{\hat{x}_{i}^{2}}{2}, \hat{p}_{i}\bigg] = 
    \frac{1}{2 i \hbar} \big(\hat{x}_{i} [\hat{x}_{i}, \hat{p}_{i}] + [\hat{x}_{i}, \hat{p}_{i}] \hat{x}_{i}\big) =
    \frac{1}{2 i \hbar} \big(\hat{x}_{i} i \hbar + i \hbar \hat{x}_{i}\big) = \hat{x}_{i}
\end{equation*}

 $\mathcal{O}(p) = \hat{p}$ is very similar.

\begin{equation*}
    \mathcal{O}(p_{i}) = \frac{1}{i \hbar}\bigg[\hat{x}_{i}, \frac{\hat{p}_{i}^{2}}{2}\bigg] = 
    \frac{1}{2 i \hbar} \big([\hat{x}_{i}, \hat{p}_{i}] \hat{p}_{i} + \hat{p}_{i} [\hat{x}_{i}, \hat{p}_{i}]\big) =
    \frac{1}{2 i \hbar} \big(i \hbar \hat{p}_{i} + \hat{p}_{i} i \hbar \big) = \hat{p}_{i}
\end{equation*}

TODO: Groenewold theorem, Deformation quantization and geometric quantization note.

From now, we will use the explicit form of the operators in the \textit{x-representation}.

\begin{equation}
    \begin{gathered}
        \mathcal{O}(p_{i}) = - i \hbar \partialder{}{x^{i}} \\
        \mathcal{O}(E) = i \hbar \partialder{}{t}
    \end{gathered}
\end{equation}


\section{Klein-Gordon equation}

\paragraph{} \textit{Klein-Gordon equation} is relativistic quantum mechanical equation named after Oskar Klein and Walter Gordon and
it describes relativistic spinless particles. It can be found using quantization of the relativistic energy equation. We need to apply the
$\mathcal{O}$ map on each side while taking into account that $\mathcal{O}(E^2) = - \hbar^{2} \partialdernd{}{t}$ and 
$\mathcal{O}(\vec{p}^2) = - \hbar^{2} \Delta$.

\begin{equation*}
    \begin{gathered}
        E^{2} = \vec{p}^{2} c^{2} + m^{2} c^{4} \\
        - \hbar^{2} \partialdernd{}{t} \psi = - \hbar^{2} c^{2} \Delta \psi + m^{2} c^{4} \psi
    \end{gathered}
\end{equation*}

After a little rearranging and setting $\frac{m^{2} c^{2}}{\hbar^{2}} = \mu^{2}$, we will finaly get the famous form \ref{eq:klein_gordon}.

\begin{equation}
    \label{eq:klein_gordon}
    \bigg(\frac{1}{c^{2}} \partialdernd{}{t} - \Delta + \mu^{2} \bigg) \psi = 0
\end{equation}

Let's rewrite the equation in a form so we can see it is \textit{lorentz invariant}. We use the fact that \textit{4-gradient}
transforms the same way as any other 4-vector and than express the Klein-Gordon equation using 4-gradient dot products.
Using the metric $\eta^{\mu \nu} = diag(1, -1, -1, -1)$ and the \textit{4-gradient} $\partial_{\mu} = (\partialder{}{t}, \partialder{}{x}, \partialder{}{y}, \partialder{}{z})$ 
it is possible to write the term $\frac{1}{c^{2}} \partialdernd{}{t} - \Delta$ as follows.

\begin{equation*}
     \frac{1}{c^{2}} \partialdernd{}{t} - \partialdernd{}{x} - \partialdernd{}{y} - \partialdernd{}{z} = \eta^{\mu \nu} \partial_{\mu} \partial_{\nu} = \partial_{\mu} \partial^{\mu}
\end{equation*}

\begin{equation}
    \bigg(\partial_{\mu} \partial^{\mu} + \mu^{2} \bigg) \psi = 0
\end{equation}

\section{Dirac equation}

\paragraph{} Dirac tried to find an equation that is first order in both space and time by doing the \textit{square root} 
of the Klein-Gordon equation. Let's start with the equation \ref{eq:klein_gordon}.

\begin{equation*}
    \bigg(- \frac{1}{c^{2}} \partialdernd{}{t} + \Delta - \mu^{2} \bigg) \psi = 0
\end{equation*}

We would like to find an operator $\hat{K}$ such that $\hat{K} \hat{K} = (-\frac{1}{c^{2}} \partialdernd{}{t} + \Delta - \mu^{2})$. Let's
find the $\hat{K}$ in terms of coefficients $\gamma^{0}$, $\gamma^{1}$, $\gamma^{2}$, $\gamma^{3}$. If we write $\hat{K}^{2} = (\hat{K}' + \mathbb{I} \mu)(\hat{K}' - \mathbb{I} \mu)$
we see it is sufficient to investigate the $\hat{K}'$ operator only since $(\hat{K}' + \mathbb{I} \mu)(\hat{K}' - \mathbb{I} \mu) = \big(\hat{K}'\big)^{2} - \mu^{2}$.

\begin{equation*}
    \frac{1}{c^{2}} \partialdernd{}{t} - \Delta = 
    (\gamma^{0} \frac{i}{c} \partialder{}{t} + \gamma^{1} \partialder{}{x} + \gamma^{2} \partialder{}{y} + \gamma^{3} \partialder{}{z})
    (\gamma^{0} \frac{i}{c} \partialder{}{t} + \gamma^{1} \partialder{}{x} + \gamma^{2} \partialder{}{y} + \gamma^{3} \partialder{}{z})
\end{equation*}

and by rewriting the expression above we get

\begin{equation*}
    \begin{gathered}
        (\gamma^{0} \frac{i}{c} \partialder{}{t} + \gamma^{1} \partialder{}{x} + \gamma^{2} \partialder{}{y} + \gamma^{3} \partialder{}{z})
        (\gamma^{0} \frac{i}{c} \partialder{}{t} + \gamma^{1} \partialder{}{x} + \gamma^{2} \partialder{}{y} + \gamma^{3} \partialder{}{z}) = \\
        \frac{1}{c^{2}} \partialdernd{}{t} (\gamma^{0})^{2} + \partialdernd{}{x} (\gamma^{1})^{2} + \partialdernd{}{y} (\gamma^{2})^{2} + \partialdernd{}{z} (\gamma^{3})^{2} +
        \partialder{}{t} \partialder{}{x} (\gamma^{0} \gamma^{1} + \gamma^{1} \gamma^{0}) + \partialder{}{t} \partialder{}{y} (\gamma^{0} \gamma^{2} + \gamma^{2} \gamma^{0}) \dots
    \end{gathered}
\end{equation*}

We see the expansion gave us the conditions for the $\gamma$'s, $(\gamma^{0})^{2} = 1$, $(\gamma^{i})^{2} = -1$ and $\gamma^{i} \gamma^{j} + \gamma^{j} \gamma^{i} = 0$
for $i, j \in \{1, 2, 3\}$. To satisfy all these conditions $\gamma^{\mu}$ is clearly not a number. When Dirac was deriving the equation
he immediately recongnised the gammas to be 4x4 matrices.

\begin{equation*}
    \begin{gathered}
        \gamma^{0} = 
         \begin{pmatrix}
          1 & 0 & 0 & 0 \\
          0 & 1 & 0 & 0 \\
          0 & 0 & -1 & 0 \\
          0 & 0 & 0 & -1 \\
         \end{pmatrix} \\
        \gamma^{1} = 
         \begin{pmatrix}
          0 & 0 & 0 & 1 \\
          0 & 0 & 1 & 0 \\
          0 & -1 & 0 & 0 \\
          -1 & 0 & 0 & 0 \\
         \end{pmatrix} \\
        \gamma^{2} = 
         \begin{pmatrix}
          0 & 0 & 0 & -i \\
          0 & 0 & i & 0 \\
          0 & i & 0 & 0 \\
          -i & 0 & 0 & 0 \\
         \end{pmatrix} \\
        \gamma^{3} = 
         \begin{pmatrix}
          0 & 0 & 1 & 0 \\
          0 & 0 & 0 & -1 \\
          -1 & 0 & 0 & 0 \\
          0 & 1 & 0 & 0 \\
         \end{pmatrix} \\
    \end{gathered}
\end{equation*}

The the final famous form of the Dirac equation is

\begin{equation}
    \label{eq:dirac_original}
    ( i \hbar \gamma^{\mu} \partial_{\mu} + mc) \psi = 0
\end{equation}
