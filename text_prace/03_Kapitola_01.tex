\chapter{Relativistic quantum mechanics}

In this chapter (TODO)

\section{Canonical quantization}

\paragraph{} Canonical transformation is a process of transition from a classical theory to
the quantum one, i.e. procedure for quantizing the theory. The term \textit{canonical} comes
from (TODO).

\begin{definition}
    \label{df:poisson}
    Given two functions $f(\vec{q}, \vec{p}, t)$ and $g(\vec{q}, \vec{p}, t)$, the binary operation
    $\{f, g\}$ is called the \textbf{Poisson bracket} of these functions and takes the form

    \begin{equation}
        \{f, g\} = \sum_{i=1}^{N}\left(\partialder{f}{q_{i}} \partialder{g}{p_{i}} - \partialder{f}{p_{i}} \partialder{g}{q_{i}}\right)
    \end{equation}
\end{definition}

\begin{theorem}
    \label{th:poisson_cononical_coordinates}
    The Poisson brackets of canonical coordinates $q_{i}$ and $p_{i}$ are

    \begin{equation}
        \begin{gathered}
        \{q_{i}, q_{j}\} = 0 \\
        \{p_{i}, p_{j}\} = 0 \\
        \{q_{i}, p_{j}\} = \delta_{ij}
        \end{gathered}
    \end{equation}
\end{theorem}

\begin{proof}
    The proof of \ref{th:poisson_cononical_coordinates} can be easily done from the definition \ref{df:poisson}. 
    The first two equations are manifestly zero because $\partialder{q_{i}}{p_{j}} = \partialder{p_{i}}{q_{j}} = 0$
    and the last can one be proved as follows.

    \begin{equation}
        \begin{split}
            \{q_{i}, p_{j}\} = \sum_{k=1}^{N}\left(\partialder{q_{i}}{q_{k}} \partialder{p_{j}}{p_{k}} - \partialder{q_{i}}{p_{k}} \partialder{p_{j}}{q_{k}}\right) 
            = \sum_{k=1}^{N} \partialder{q_{i}}{q_{k}} \partialder{p_{j}}{p_{k}} = 
            \begin{cases}
                1  & \quad \text{if } i = j = k \\
                0  & \quad \text{otherwise}
            \end{cases} \\
            \implies \{q_{i}, p_{j}\} = \delta_{ij}
        \end{split}
    \end{equation}
\end{proof}


    
\section{Klein-Gordon equation}

\section{Dirac equation}

