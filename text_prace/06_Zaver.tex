\chapter*{Conclusion}
\addcontentsline{toc}{chapter}{Conclusion}

\paragraph{} In the thesis we tried to demonstrate obstacles of the \textit{first quantization}, namely
problems we have to deal with when we want a relativistic quantum theory. By showing the proof of Groenewold's theorem
we ended with the conclusion the original quantization map proposed by Dirac actually doesn't exists and one have
to look for more sophisticated methods such as \textit{deformation quantization}. While it is totally out of scope 
of the thesis it might be interesting to note that a lot of articles try to face the problem of quantization using
the language of category theory. Then we continued by constructing the Klein-Gordon equation and Dirac equation
which were attempts to get a relativistic form of the Schrödinger equation.

\paragraph{} In the second chapter we wanted to discuss what problems we face with the relativistic quantum mechanics.
Firstly, we saw that in the case of the Klein-Gordon equation we have a problem with the probabilistic 
interpretation of the wave function since the continuity equation allowed negative probability densities.
Then, we checked that by introducing the Dirac equation this problem is really solved and in that case 
the probability density is positive definite. In the second part of the chapter we focused on the
causality in physics. The investigation started with the condition for testing whether the theory does
break causality. Then we calculated both numerically and analytically using the steepest descent method
the behavior of the amplitude for a free relativistic particle near the cover of the light cone and we
found the amplitude is non-zero even for the space-time position outside of the light-cone. If such a fundamental
principle like the causality breaks it clearly indicates a conceptual problem of the theory.

\paragraph{} In the last chapter we sketched some basic principles in the quantum theory of the free scalar field.
Firstly, we briefly introduced the classical Klein-Gordon field and checked it takes the form of a combination
of waves. Then we introduced the field operator using the creation an annihilation operators and checked the
commutation relation for the field operators. Finally, we discussed again the condition for the causality
and found it is reasonable to calculate the commutator of the field operator for different space-time points
and check it must vanish for the space-like points, otherwise causality is clearly violated. Then using the argument 
of the Lorentz transformation outside of the light cone we showed the commutator really vanishes outside of the light 
cone.
