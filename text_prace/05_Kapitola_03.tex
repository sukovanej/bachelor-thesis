\chapter{Quantum field theory}

In the first chapter we tried to make a smooth transition from the "classical" quantum mechanics
to the relativistic one by considering the rules of the \textit{first quantization}. Our motivation
was to use the notations from the old theory, i.e. remain with the wave function being the
mathematical object representing the physical state and operators corresponding to the dynamical 
variables. As was shown in the second chapter using this approach we must deal a number of
various conceptual problems like negative probabilities or the break of causality. 

\paragraph{} Let's start with an observation. Using the canonical quantization we replaced each momentum, position
and energy variable with the corresponding operator. So there is an operator for the position
but not a single word was mentioned about the time. Why do we have a position operator but not a
time operator? The time in all the equations was just a parameter. This asymmetry between the time
and space seems to be an another nail in the coffin of the first quantization. When searching for
the correct relativistic quantum theory we should seek such objects that treat space-time variables
on the same footing. 

\section{Scalar fields}

\paragraph{} Let's start with the classical \textit{Lagrangian} function $L$. Lagrangian $L$ is being defined as a 
function $L(q_{1}, \dots, q_{n}, \dot{q}_{1}, \dots, \dot{q}_{n}, t)$ describing the evolution of 
$(q_{1}(t), \dots, q_{n}(t), \dot{q}_{1}(t), \dots, \dot{q}_{n}(t))$ in the configuration space using the least action 
principle $\delta S = 0$,

\begin{equation}
    \label{eq:action}
    S = \int dt \ L(q_{1}(t), \dots, q_{n}(t), \dot{q}_{1}(t), \dots, \dot{q}_{n}(t))
\end{equation}

Now we want to investigate the dynamics of a field $\phi(\vec{x}, t)$ using the \textit{Lagrangian density}
$\mathcal{L}(\phi, \partialder{\phi}{\vec{x}}, \partialder{\phi}{t}, \dots)$ instead.

\begin{equation}
    \label{eq:lagrangian_density}
    L = \int d\vec{x} \ \mathcal{L}(\phi, \partialder{\phi}{\vec{x}}, \partialder{\phi}{t}, \partialdernd{\phi}{\vec{x}}, \partialdernd{\phi}{t} \dots)
\end{equation}

From now, we will deal with time $t$ and space $\vec{x}$ using the convenient relativistic notation

\begin{equation*}
    x^{\mu} = (ct, x^{1}, x^{2}, x^{3})
\end{equation*}
    
and the 4-gradient is given by 

\begin{equation*}
    \partial_{\mu} = (\frac{1}{c} \partialder{}{t}, \partialder{}{x^{1}}, \partialder{}{x^{2}}, \partialder{}{x^{3}})
\end{equation*}

As was shown in \cite{goldstein} using the Lagrangian $\mathcal{L} = \mathcal{L}(\phi, \partial_{\mu} \phi)$ the dynamics of $\phi$ is given 
by the equation

\begin{equation}
    \label{eq:euler_lagrange}
    \partialder{\mathcal{L}}{\phi} - d_{\mu} \bigg[ \partialder{\mathcal{L}}{(\partial_{\mu} \phi)} \bigg] = 0
\end{equation}

Now we will investigate a simple example of the Lagrangian which will come in use later. Also, since it is a standard in
the literature to use the \textit{natural units} we will also set $\hbar = 1$ and $c = 1$ and if needed the constants
can be retrieved using the dimensional analysis.

\begin{equation}
    \label{eq:klein_gordon_field}
    \mathcal{L} = \frac{1}{2} \partial_{\mu} \phi \partial^{\mu} \phi - \frac{1}{2} m^{2} \phi^{2}
\end{equation}

Let's find the equation for the Lagrangian \ref{eq:klein_gordon_field} using \ref{eq:euler_lagrange}.

\begin{equation*}
    \begin{gathered}
        \partialder{\mathcal{L}}{\phi} = - m^2 \phi \\
        \frac{d}{dx^{\mu}} \bigg[ \partialder{\mathcal{L}}{(\partial_{\mu} \phi)} \bigg] = \partial^{\mu} \partial_{\mu} \phi
    \end{gathered}
\end{equation*}

The equation for the field $\phi$ takes the form \ref{eq:klein_gordon_field_equation} and is very similar to the Klein-Gordon
equation from the first chapter!

\begin{equation}
    \label{eq:klein_gordon_field_equation}
    \partial^{\mu} \partial_{\mu} \phi + m^2 \phi = 0
\end{equation}

Let's calculate the Fourier transformation of $\phi$ and the Klein-Gordon equation.

\begin{equation}
    \phi (\vec{x}, t) = \int_{\mathbb{R}^{3}} \frac{dp^{3}}{(2 \pi)^{3}} \ e^{i \vec{p} \vec{x}} \ \phi (\vec{p}, t)
\end{equation}

The equation \ref{eq:klein_gordon_field_equation} then becomes

\begin{equation*}
    \begin{gathered}
        (\partial^{\mu} \partial_{\mu} + m^2) \int_{\mathbb{R}^{3}} \frac{dp^{3}}{(2 \pi)^{3}} \ e^{i \vec{p} \vec{x}} \ \phi (\vec{p}, t) = 0 \\
        \int_{\mathbb{R}^{3}} \frac{dp^{3}}{(2 \pi)^{3}} \ (\partial_{t}^{2} - \Delta + m^2) \ e^{i \vec{p} \vec{x}} \ \phi (\vec{p}, t) = 0 \\
        \int_{\mathbb{R}^{3}} \frac{dp^{3}}{(2 \pi)^{3}} \ (\partial_{t}^{2} + | \vec{p} |^{2} + m^2) \ e^{i \vec{p} \vec{x}} \ \phi (\vec{p}, t) = 0
    \end{gathered}
\end{equation*}

which results in the equation

\begin{equation*}
    (\partial_{t}^{2} + | \vec{p} |^{2} + m^2) \phi (\vec{p}, t) = 0
\end{equation*}

Thus we see the dispersion relation is $E^{2} = | \vec{p} |^{2} + m^2$. We should seek the solution in form of combination of $e^{i (E t - \vec{p} \vec{x})}$
and $e^{- i (E t - \vec{p} \vec{x})}$.

\begin{equation*}
    \phi (\vec{x}, t) = \int_{\mathbb{R}^{3}} \frac{dp^{3}}{(2 \pi)^{3} \sqrt{2 E_{p}}} \ \bigg[a e^{i (p_{0} t - \vec{p} \vec{x})} + a' e^{- i (p_{0} t - \vec{p} \vec{x})}\bigg]
\end{equation*}

In the correspondence with the classical Lagrangian, Hamiltonian and momentum, we can also define Hamiltonian density and "momentum density". Let's 
denote $\dot{\phi} = \partial_{t} \phi$. 

\begin{equation*}
    \begin{gathered}
        \Pi(\vec{x}, t) = \partialder{\mathcal{L}}{\dot{\phi}} \\
        \mathcal{H} = \Pi(\vec{x}, t) \cdot \dot{\phi}(\vec{x}, t) - \mathcal{L}
    \end{gathered}
\end{equation*}

In the case of the free scalar field \ref{eq:klein_gordon_field} the $\Pi$ is given by

\begin{equation*}
    \begin{gathered}
        \Pi(\vec{x}, t) = \partialder{\mathcal{L}}{\dot{\phi}} = \partialder{}{\dot{\phi}} \bigg[\frac{1}{2} \partial_{\mu} \phi \partial^{\mu} \phi - \frac{1}{2} m^{2} \phi^{2}\bigg]
        = \dot{\phi}
    \end{gathered}
\end{equation*}

which we can use the see the form of the Hamiltonian density.

\begin{equation*}
    \begin{gathered}
        \mathcal{H} = \dot{\phi}^{2} - \frac{1}{2}\bigg[\partial_{\mu} \phi \partial^{\mu} \phi - m^{2} \phi^{2}\bigg] = 
        \dot{\phi}^{2} - \frac{1}{2}\bigg[\dot{\phi}^{2} - \big(\nabla \phi\big)^{2} - m^{2} \phi^{2}\bigg] = \\
        \frac{1}{2}\bigg[\dot{\phi}^{2} + \big(\nabla \phi\big)^{2} + m^{2} \phi^{2}\bigg]
    \end{gathered}
\end{equation*}

\section{Second quantization}

\paragraph{} In the procedure of second quantization we must start with the representation of the state of the system. We
will use the \textit{occupation number representation} which counts the number of particles in each state. We
will write the \textit{vacuum state} as $\ket{0} = \ket{\dots, 0, \dots}$. The general state $\ket{n_{1}, \dots, n_{N}}$
represents the system with $n_{1}$ particles in the first state, etc. In the analogy with the harmonic oscillator we will
introduce \textit{creation} and \textit{annihilation} operators $\hat{a}_{\vec{p}}$ and $\hat{a}^{\dagger}_{\vec{p}}$.
The basic properties are that $\hat{a}_{\vec{p}}$ annihilates the vacuum state and $\hat{a}^{\dagger}_{\vec{p}}$ creates
the $\ket{\vec{p}}$ state.

\begin{equation*}
    \begin{gathered}
        \hat{a}_{\vec{p}} \ket{0} = 0 \\
        \hat{a}^{\dagger}_{\vec{p}} \ket{0} = \ket{\vec{p}}
    \end{gathered}
\end{equation*}

The commutation relations are

\begin{equation*}
    \begin{gathered}
        \big[\hat{a}_{\vec{p}}, \hat{a}_{\vec{p}'}\big] = \big[\hat{a}_{\vec{p}}^{\dagger}, \hat{a}_{\vec{p}'}^{\dagger}\big] = 0 \\
        \big[\hat{a}_{\vec{p}}, \hat{a}_{\vec{p}'}^{\dagger}\big] = (2 \pi)^{3} \delta(\vec{p} - \vec{p}')
    \end{gathered}
\end{equation*}

\paragraph{} In this place I would like to cite a nice recipe from \cite{gifted_amateur} which summarizes
the canonical quantization we will use for going from a classical field theory to the quantum 
field theory:

\begin{itemize}
    \item \textbf{Step I}: Write down a classical Lagrangian density in terms of fields. This is the creative part because there are lots of 
        possible Lagrangians. After this step, everything else is automatic.
    \item \textbf{Step II}: Calculate the momentum density and work out the Hamiltonian density in terms of fields.
    \item \textbf{Step III}: Now treat the fields and momentum density as operators. Impose commutation relations on them to make them 
        quantum mechanical.
    \item \textbf{Step IV}: Expand the fields in terms of creation/annihilation operators. This will allow us to use occupation numbers 
        and stay sane.
    \item \textbf{Step V}: That's it. Congratulations, you are now the proud owner of a working quantum field theory, provided you remember 
        the normal ordering interpretation.
\end{itemize}

Let's rewrite the Klein-Gordon field with the creation and annihilation operator to promote the $\phi$ field to the field operator. We expect
the $\phi$ and $\Pi$ field to have commutation relation in the same form as they had in the discrete case. But instead of discrete $\delta^{\mu}_{\nu}$ matrix
we should use the Dirac delta function.

\begin{equation*}
    \begin{gathered}
        \big[\phi(x), \phi(x')] = \big[\Pi(x), \Pi(x')] = 0 \\
        \big[\phi(x), \Pi(x')] = i \delta(x - x')
    \end{gathered}
\end{equation*}

The first one will be shown in the next section for the case of a time-like space-time events. The second one can be calculated
directly using the scalar field obtained in the previous part. But we will replace the constants $a_{p} \to \hat{a}_{p}$, 
$a'_{p} \to \hat{a}^{\dagger}_{p}$ and check whether the choice is fine by the commutator value.

\begin{equation*}
    \begin{gathered}
        \big[\phi(x), \Pi(x')] = \\
        \bigg[
            \int_{\mathbb{R}^{3}} \frac{dp^{3}}{(2 \pi)^{3} \sqrt{2 E_{p}}} \ \big(\hat{a}_{p} e^{i p x} + \hat{a}^{\dagger}_{p} e^{- i p x}\big),
            \int_{\mathbb{R}^{3}} \frac{dp'^{3}}{(2 \pi)^{3}} (-i) \sqrt{\frac{E_{p'}}{2}} \ \big(\hat{a}_{p'} e^{i p' x'} - \hat{a}^{\dagger}_{p'} e^{- i p' x'}\big)
        \bigg] = \\
    \end{gathered}
\end{equation*}

There is a little trick that we replace $\hat{a}^{\dagger}_{p} e^{- i p x}$ by $\hat{a}^{\dagger}_{-p} e^{i p x}$. After this we will
use the commutation relation for $\hat{a}_{p}$ and $\hat{a}^{\dagger}_{p}$ to simplify the commutator significantly.

\begin{equation*}
    \begin{gathered}
        \iint_{\mathbb{R}^{3} \cross \mathbb{R}^{3}} \frac{dp^{3} dp'^{3}}{(2 \pi)^{6}} \frac{(-i)}{2} \sqrt{\frac{E_{p'}}{E_{p}}}
        \bigg[
            \hat{a}_{p} e^{i p x} + \hat{a}^{\dagger}_{-p} e^{i p x},
            \hat{a}_{p'} e^{i p' x'} - \hat{a}^{\dagger}_{-p'} e^{i p' x'}
        \bigg] = \\
        \iint_{\mathbb{R}^{3} \cross \mathbb{R}^{3}} \frac{dp^{3} dp'^{3}}{(2 \pi)^{6}} \frac{(-i)}{2} \sqrt{\frac{E_{p'}}{E_{p}}}
        \bigg(
           - \big[ \hat{a}_{p'}, \hat{a}^{\dagger}_{-p} \big]
           - \big[ \hat{a}_{p}, \hat{a}^{\dagger}_{-p'} \big] 
        \bigg) e^{i p' x' + i p x}=
    \end{gathered}
\end{equation*}

Finally, let's replace the commutators with the delta functions $\big[\hat{a}_{p}, \hat{a}^{\dagger}_{p'} \big] = (2 \pi)^{3} \delta(p - p')$.

\begin{equation*}
    \begin{gathered}
        \iint_{\mathbb{R}^{3} \cross \mathbb{R}^{3}} \frac{dp^{3} dp'^{3}}{(2 \pi)^{3}} \frac{i}{2} \sqrt{\frac{E_{p'}}{E_{p}}}
        \bigg( \delta(p' + p) + \delta(p' + p) \bigg) e^{i p' x' + i p x} = \\
        \int_{\mathbb{R}^{3}} \frac{dp^{3}}{(2 \pi)^{3}} i e^{- i p(x' - x)} = i \delta(x - x') \\
    \end{gathered}
\end{equation*}

It is also worth discussing the factor $(2 E_{p})^{\frac{1}{2}}$ in the $\phi$. The idea is
to use the dispersion relation $p = (p_{0}, \vec{p}) \implies p^{2} - m^2 = 0$ to solve the problem that $dp^{3}$ is
not Lorentz invariant. We choose $dp^{4} \delta(p^{2} - m^2) \theta(p_{0})$ as the measure which takes into account 
the \textit{mass shell condition} introduced by the dispersion relation. Using the identity

\begin{equation*}
    \delta(f(x)) = \sum_{\{x_{0} \\ | \\ f(x_{0}) = 0\}} \frac{1}{|f'(x_{0})|} \delta(x - x_{0})
\end{equation*}

for $E_{p} = \sqrt{\vec{p}^{2} + m^{2}}$ we can show $\delta(p^{2} - m^2) \theta(p_{0}) = \frac{1}{2 E_{p}} (\delta(p_{0} - E_{p}) + 
\delta(p_{0} + E_{p})) \theta(p_{0}) = \frac{1}{2 E_{p}} \delta(p_{0} - E_{p}) \theta(p_{0})$. Using the new measure which
the identity operator is expressed as 

\begin{equation*}
    \mathbb{I}_{p} = \int_{\mathbb{R}} \frac{dp^{3}}{(2 \pi)^{3} 2 E_{p}} \ket{p} \bra{p}
\end{equation*}

and if we try to calculate $\bra{p} \mathbb{I} \ket{q}$ we expect this to be a Dirac delta $\delta(p - q)$.

\begin{equation*}
    \bra{p} \mathbb{I}_{p} \ket{q} = \int_{\mathbb{R}} \frac{dp^{3}}{(2 \pi)^{3} 2 E_{p}} \bra{p} \ket{p} \bra{p}\ket{q} = \frac{1}{(2 \pi)^{3} 2 E_{p}} \delta(p - q)
\end{equation*}

So we will rather normalize the momentum state $\ket{\tilde{p}} = (2 \pi)^{\frac{3}{2}} (2 E_{p})^{\frac{1}{2}} \ket{p}$ and
the annihilation and creation operators $\ket{\tilde{p}} = \tilde{a}^{\dagger} \ket{0} = (2 \pi)^{\frac{3}{2}} (2 E_{p})^{\frac{1}{2}} \ket{p}$.
The field operator using the measure and new $\tilde{a}$ operators is as follows.

\begin{equation*}
    \int_{\mathbb{R}^{3}} \frac{dp^{3}}{(2 \pi)^{3} 2 E_{p}} \ \big(\tilde{a}_{p} e^{i p x} + \tilde{a}^{\dagger}_{p} e^{- i p x}\big) = 
    \int_{\mathbb{R}^{3}} \frac{dp^{3} (2 \pi)^{\frac{3}{2}} (2 E_{p})^{\frac{1}{2}}}{(2 \pi)^{3} 2 E_{p}} \ \big(\tilde{a}_{p} e^{i p x} + \tilde{a}^{\dagger}_{p} e^{- i p x}\big) = 
\end{equation*}

\section{Causality problem}

Now we shall check whether the new formulation for a free scalar relativistic theory gives the right results about causality.
The first thing we could do is basically calculate the propagator $\bra{0} \hat{\phi}(x) \hat{\phi}(y) \ket{0}$ and hope the 
value will be zero outside of the light cone. But, instead we should go back and check whether our condition for the causality
is still valid in the case of the quantum field theory. We know that if two observables $A$ and $B$ are being measured we can
decide whether the measurement of $A$ influences the value of $B$ by calculating their commutator $[A, B]$. It is important that
if $[A, B] \neq 0$ then by measuring $A$ it is possible to decide whether $B$ has bee measured on the system or not and vice 
versa.

This means that when $[A, B] \neq 0$ the event must be time-like\footnote{
\textit{Time-like} separated events are inside of the light cone of each other.
\textit{Space-like} separated events are outside of the light cone of each other and thus the Lorentz transformation of those 
events does not preserve time ordering.} since they are \textit{correlated}. If $[A, B] = 0$ then there is no information of possible
measurement $B$ when doing the measurement $A$ and thus the event might be space-like and it doesn't matter we could do a transformation
into a frame such that we reverse the order of the events.

\begin{definition}
    \label{df:causality_field_theory}
    Causality is not preserved if for a space-like separated points $x$, $y$ the field operator doesn't commute with itself in these points, i.e.
    we can simultaneously measure values which are correlated and the time ordering is not preserved for different frames. Technically for
    $(x - y) \eta (x - y) < 0$

    \begin{equation}
        \big[\phi(x), \phi(y)\big] \neq 0
    \end{equation}

\end{definition}

So to check the causality we will check the commutator $[\phi(x), \phi(y)]$.

\begin{equation*}
    \begin{gathered}
        \bigg[\phi(x), \phi(x')\bigg] = \\
        \bigg[
            \int_{\mathbb{R}^{3}} \frac{dp^{3}}{(2 \pi)^{3}} \frac{1}{\sqrt{2 E_{p}}} \big(\hat{a}_{p} e^{ipx} + \hat{a}^{\dagger}_{p} e^{-ipx}\big),
            \int_{\mathbb{R}^{3}} \frac{dp'^{3}}{(2 \pi)^{3}} \frac{1}{\sqrt{2 E_{p'}}} \big(\hat{a}_{p'} e^{ip'x'} + \hat{a}^{\dagger}_{p'} e^{-ip'x'}\big)
        \bigg] = \\
        \iint_{\mathbb{R}^{3} \cross \mathbb{R}^{3}} \frac{dp^{3} \ dp'^{2}}{(2 \pi)^{6}} \frac{1}{2 \sqrt{E_{p} E_{p'}}}
        \bigg[
            \hat{a}_{p} e^{ipx} + \hat{a}^{\dagger}_{p} e^{-ipx},
            \hat{a}_{p'} e^{ip'x'} + \hat{a}^{\dagger}_{p'} e^{-ip'x'}
        \bigg] = \\
        \iint_{\mathbb{R}^{3} \cross \mathbb{R}^{3}} \frac{dp^{3} \ dp'^{2}}{(2 \pi)^{6}} \frac{1}{2 \sqrt{E_{p} E_{p'}}}
        \bigg(
            \big[\hat{a}_{p} e^{ipx}, \hat{a}^{\dagger}_{p'} e^{-ip'x'}\big] + \big[\hat{a}^{\dagger}_{p} e^{-ipx}, \hat{a}_{p'} e^{ip'x'}\big]
        \bigg) = \\
        \iint_{\mathbb{R}^{3} \cross \mathbb{R}^{3}} \frac{dp^{3} \ dp'^{2}}{(2 \pi)^{3}} \frac{1}{2 \sqrt{E_{p} E_{p'}}}
        \big(\delta(p - p') e^{ipx - ip'x'} - \delta(p - p') e^{-(ipx - ip'x')} \big) = \\
        \int_{\mathbb{R}^{3}} \frac{dp^{3}}{(2 \pi)^{3}} \frac{1}{2 E_{p'}} \big(e^{ip(x - x')} - e^{-ip(x - x')} \big)
    \end{gathered}
\end{equation*}

Now, as discussed in \cite{peskin_schroeder}, there is the trick doing the Lorentz transformation of the second term $-(x - y) \to (x - y)$\footnote{Clearly $-\mathbb{I}$ 
is the desired inversion $x^{\mu} \to -x^{\mu} = (-\delta^{\mu}_{\nu}) x^{\nu}$ and it is a Lorentz transformation because $(-\mathbb{I}) \eta 
(-\mathbb{I}^{T}) = \eta$. Also we can see geometrically Lorentz transformation is moving the point on the sphere of the hyperboloid outside of the
light cone but there is no continuous map between the hyperbolas in the future light cone and the past cone.} for case $x -  y$ in the case the $x - y$ is space-like. Since it
is a Lorentz invariant, terms in the integral cancel each other and thus the commutator is zero. Inside of the light cone, we don't have such a continuous transformation
because we would have to map a point from the future light cone to the past light cone. Because of that the commutator is non-zero for the time-like separated points which means 
inside of the light cone events might be causal.

\begin{figure}[H]
    \centering
    \includegraphics[width=0.5\textwidth]{lorentz_transform.png}
    \caption{Lorentz transformation of the space-like event.}
    \label{fig:lorentz_transform}
\end{figure}

