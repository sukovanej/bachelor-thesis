\chapter{Quantum field theory}

In the first chapter we tried to make a smooth transition from the "classical" quantum mechanics
to the relativistic one by considering the rules of the \textit{first quantization}. Our motivation
was to use preserve the notations from the old theory, i.e. remain with the wavefunction being the
mathematical object representing the physical state and operators corresponding to the dynamical 
variables. As was shown in the second chapter using this approach we must deal a number of
various conceptual problems like negative probabilities or the break of causality. 

\paragraph{} Let's start with an observation. Using the canonical quantization we replaced each momentum, position
and energy variable with the corresponding operator. So there is an operator for the position
but not a single word was mentioned about the time. Why do we have a position operator but not a
time operator? The time in all the equations was just a parameter. This assymetry between the time
and space seems to be an another nail in the coffin of the first quantization. When searching for
the the correct relativistic quantum thery we should seek such objects that treat space-time variables
on the same footing. 

\section{Scalar fields}

\paragraph{} Let's start with the classical \textit{Lagrangian} function $L$. Langragian $L$ is being defined as a 
function $L(q_{1}, \dots, q_{n}, \dot{q}_{1}, \dots, \dot{q}_{n}, t)$ describing the evolution of 
$(q_{1}(t), \dots, q_{n}(t), \dot{q}_{1}(t), \dots, \dot{q}_{n}(t))$ in the configuration space using the least action 
principle $\delta S = 0$,

\begin{equation}
    \label{eq:action}
    S = \int dt \ L(q_{1}(t), \dots, q_{n}(t), \dot{q}_{1}(t), \dots, \dot{q}_{n}(t))
\end{equation}

Now we want to investigate the dynamics of a field $\phi(\vec{x}, t)$ using the \textit{Lagrangian density}
$\mathcal{L}(\phi, \partialder{\phi}{\vec{x}}, \partialder{\phi}{t}, \dots)$ instead.

\begin{equation}
    \label{eq:lagrangian_density}
    L = \int d\vec{x} \ \mathcal{L}(\phi, \partialder{\phi}{\vec{x}}, \partialder{\phi}{t}, \partialdernd{\phi}{\vec{x}}, \partialdernd{\phi}{t} \dots)
\end{equation}

From now, we will deal with time $t$ and space $\vec{x}$ using the convenient relativistic notation

\begin{equation*}
    x^{\mu} = (ct, x^{1}, x^{2}, x^{3})
\end{equation*}
    
and the 4-gradient is given by 

\begin{equation*}
    \partial_{\mu} = (\frac{1}{c} \partialder{}{t}, \partialder{}{x^{1}}, \partialder{}{x^{2}}, \partialder{}{x^{3}})
\end{equation*}

As was shown in \cite{goldstein} using the Lagrangian $\mathcal{L} = \mathcal{L}(\phi, \partial_{\mu} \phi)$ the dynamics of $\phi$ is given 
by the equation

\begin{equation}
    \label{eq:euler_lagrange}
    \partialder{\mathcal{L}}{\phi} - \partial_{\mu} \bigg[ \partialder{\mathcal{L}}{(\partial_{\mu} \phi)} \bigg] = 0
\end{equation}

TODO: show the lorentz invariance of the euler legrange eq

Now we will investigate a simple example of the lagrangian which will come in use later. TODO: units discussions

\begin{equation}
    \label{eq:klein_gordon_field}
    \mathcal{L} = \frac{1}{2} \partial_{\mu} \phi \partial^{\mu} \phi + \frac{1}{2} m^{2} \phi^{2}
\end{equation}

Let's find the the equation for the Lagrangian \ref{eq:klein_gordon_field} using \ref{eq:euler_lagrange}.

TODO: klein-gordon field

\begin{equation*}
    \begin{gathered}
        \partialder{\mathcal{L}}{\phi} = m^2 \phi
        \partial_{\mu} \bigg[ \partialder{\mathcal{L}}{(\partial_{\mu} \phi)} \bigg] = \partial^{\mu} \partial_{\mu} \phi
    \end{gathered}
\end{equation*}

TODO: klein-gordon solution

\section{Second quantization}

TODO: a and a dagger operators

In this place I would like to cite a nice recipe from \cite{gifted_amateur} which summarizes
the canonical quantization we will use for going from a classical field theory to the quantum 
field theory:

\begin{itemize}
    \item \textbf{Step I}: Write down a classical Lagrangian density in terms of fields. This is the creative part because there are lots of 
        possible Lagrangians. After this step, everything else is automatic.
    \item \textbf{Step II}: Calculate the momentum density and work out the Hamiltonian density in terms of fields.
    \item \textbf{Step III}: Now treat the fields and momentum density as operators. Impose commutation relations on them to make them 
        quantum mechanical.
    \item \textbf{Step IV}: Expand the fields in terms of creation/annihilation operators. This will allow us to use occupation numbers 
        and stay sane.
    \item \textbf{Step V}: That's it. Congratulations, you are now the proud owner of a working quantum field theory, provided you remember 
        the normal ordering interpretation.
\end{itemize}

TODO: quantize klein-gordon field

\section{Causality problem}

TODO: causality using the klein-gordon field
