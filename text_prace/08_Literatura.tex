\begin{thebibliography}{10}
\addcontentsline{toc}{chapter}{\bibname}

    \bibitem{sjM.kmS07}
    S.~J. Monaquel a K.~M. Schmidt, \textit{On $M$-functions and operator
      theory for non-self-adjoint discrete Hamiltonian systems}

    \bibitem{doi:10.1119/1.1934740}
        Shewell, J. R. (1959). \textit{On the Formation of Quantum-Mechanical 
        Operators. American Journal of Physics}

    \bibitem{peskin_schroeder}
        M. Peskin and D. Schroeder, \textit{An Introduction to Quantum Field Theory},
        Westview Press, Chicago, 1995.

    \bibitem{dirac_equation_history}
        G. Rajasekaran, \textit{The Discovery of Dirac Equation and its Impact on Present-day Physics},
        2003

    \bibitem{gamma_matrices}
        Wikipedie, \textit{Gamma matrices},
        [online] Available at: https://en.wikipedia.org/wiki/Gamma\_matrices, [Accessed 12 March 2019]

    \bibitem{dirac_matrices}
        Palash B. Pal, \textit{Representation-independent manipulations with Dirac matrices and spinors},
        Saha Institute of Nuclear Physics 1/AF Bidhan-Nagar, Calcutta 700064, INDIA

    \bibitem{floarin_jung}
        Florian Jung, \textit{Canonical group quantization and boundary conditions},
        Mainz Univ. (Germany). Fachbereich 08: Physik, Mathematik und Informatik

    \bibitem{imaginary_gaussian_integral}
        Howard Haber, (2018). \textit{A Gaussian integral with a purely imaginary argument}. 
        [online] Available at: http://scipp.ucsc.edu/~haber/ph215/Gaussian.pdf [Accessed 13 May 2019].

    \bibitem{gifted_amateur}
        Lancaster, T. and Blundell, S. (2018). \textit{Quantum field theory for the gifted amateur}. 
        Oxford: Oxford University Press.

    \bibitem{goldstein}
        H. Goldstein, Ch. Poole, J. Safko, \textit{Classical Mechanics}. 
        Addison Wesley
\end{thebibliography}

\cleardoublepage
