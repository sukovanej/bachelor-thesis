\chapter{Failures of the Relativistic Quantum Mechanics}

In this chapter we will discuss the problems of relativistic quantum mechanics. We will start
with the problem of probability density within the Klein-Gordon equation.

\section{Probability density}

\paragraph{} In the non-relativistic case we can find the continuity equation by taking the complex conjugate
of the Schrodinger's equation.

\begin{equation*}
    \begin{gathered}
        i \hbar \partialder{}{t} \psi = \hat{H} \psi \\
        - i \hbar \partialder{}{t} \psi^{*} = \big(\hat{H} \psi \big)^{*}
    \end{gathered}
\end{equation*}

Then we mutiply the first equation by $\psi$ and the second one by $\psi^{*}$.

\begin{equation*}
    \begin{gathered}
        i \hbar \psi^{*} \partialder{}{t} \psi = \psi^{*} \hat{H} \psi \\
        i \hbar \psi \partialder{}{t} \psi^{*} = \psi \big(\hat{H} \psi \big)^{*}
    \end{gathered}
\end{equation*}

If we substract these two equations from each other and use the formula for derivative $(a \cdot b)' = a' \cdot b + a \cdot b'$
we wil get a form from which it will be easy to identify the continuity equation $\partialder{\rho}{t} + div \vec{j} = 0$.

\begin{equation*}
    i \hbar \partialder{\left|\psi\right|^{2}}{t}  = \psi^{*} \hat{H} \psi - \psi \big(\hat{H} \psi \big)^{*}
\end{equation*}

We assume the potential $V$ in $\hat{H} = \frac{\hat{p}^{2}}{2m} + V$ to be real-valued thus the right-hand side
will simplify to

\begin{equation*}
    \psi^{*} \frac{\hbar^{2}}{2m} \nabla^{2} \psi - \psi \frac{\hbar^{2}}{2m} \nabla^{2} \psi^{*} 
    = \frac{\hbar^{2}}{2m} \bigg( \psi^{*} \nabla^{2} \psi - \psi \nabla^{2} \psi^{*} \bigg)
\end{equation*}

and the desired form is

\begin{equation*}
    \partialder{\left|\psi\right|^{2}}{t} + div \bigg[ \frac{\hbar}{2mi} \bigg( \psi \nabla \psi^{*} - \psi^{*} \nabla \psi \bigg) \bigg] = 0
\end{equation*}

Now we will do the same for the Klein-Gordon equation \ref{eq:klein_gordon}. Let's do the same procedure here and
find the complex conjugate, multiply both equations by $\psi$ and $\psi^{*}$ and then substract them. After that 
let's focus on the term which contains the time derivative a thus will represent the \textit{probability density}.

\begin{equation*}
    \partialder{\rho}{t} \sim \frac{1}{2i} \bigg[ \psi^{*} \partialdernd{}{t} \psi - \psi \partialdernd{}{t} \psi^{*} \bigg]
\end{equation*}

Now we can find what $\rho$ is proportial to. 

\begin{equation}
    \rho \sim \frac{1}{2i} \bigg[ \psi^{*} \partialder{}{t} \psi - \psi \partialder{}{t} \psi^{*} \bigg]
\end{equation}

Let's divide the wave function $\psi$ into the real and imaginary part.

\begin{equation*}
    \begin{gathered}
        \psi = i \psi_{1} + \psi_{2} \\
        \rho \sim \bigg[ \psi_{2} \partialder{}{t} \psi_{1} - \psi_{1} \partialder{}{t} \psi_{2} \bigg]
    \end{gathered}
\end{equation*}

Now it is easy to see the problem. In the non-relativistic case the probability density was simply $|\psi|^{2}$ which is
non-negative for any $\psi: \mathbb{R} \to \mathbb{C}$. In the case of Klein-Gordon equation, we can see the probability 
density is zero whenever the wave function is either exclusively real or imaginary. Moreover, since both 
$\partialder{\psi_{1}}{t}$ and $\partialder{\psi_{2}}{t}$ are arbitrary, the value can be negative. This is in contradiction
with the intepretation of $\rho$ as the \textit{probability density of measuring the system within a given state $\psi$}.

\section{Causality}

Now we would like to investigate the problem of causality in the quantum mechanics. We're going to study the transition
amplitude of a free particle to propagate from one space-time point to another one. We will assume the standard definition
of the causility in physics.

\begin{definition}
    \label{df:causality}
    Suppose we understand what is meant by a \textit{cause} $C$ and a \textit{effect} $E$. If $C$ happens before $E$ and 
    $E$ is in the future lightcone from the perspective of $C$ then there might be a causality between $C$ and $E$.  Or
    formulated more usefuly for the computation - if $X_{C} = (0, 0)$ is the $C$'s space-time event and 
    $X_{E} = (x_{E}, ct_{E})$ is the $E$'s space-time event then $X_{C}$ can cause the $X_{E}$ if $t_{E} > |x_{E}|$.
\end{definition}

We will follow the same calculation like in \cite{peskin_schroeder}. In the correspondence with the fact that the space translation
is generated by the momentum operator $\hat{p}$, the time evaluation is generated by the energy operator $\hat{H}$. So
if we know the state of the system $\ket{x_{0}} = \ket{x(t=0)}$, the state in the future will be 
$\ket{x_{0}} = \ket{x(t=t')} = \hat{U}(0, t') \ket{x_{1}} = \exp{- \frac{i}{\hbar} \hat{H} t'} \ket{x_{0}}$. The amplitude of the
system going from the state $\ket{x_{0}}$ to $\ket{x_{1}}$ is 

\begin{equation}
    \bra{x_{1}} \exp{- \frac{i}{\hbar} \hat{H} t'}\ket{x_{0}}
\end{equation}

which we will evaluate using the identity $\hat{I}$ expressed in the continuous basis $\{\ket{p}\}$ and the expression for
$\bra{p}\ket{x} = \exp{- \frac{i}{\hbar} p x}$. We will assume a single particle without an external potential thus the
Hamiltonian is simply $\hat{H} = \frac{\hat{p}^{2}}{2m}$.

\begin{equation*}
    \begin{gathered}
        A_{1} = \bra{x_{1}} e^{- \frac{i}{\hbar} \frac{\hat{p}^{2}}{2m} t'}\ket{x_{0}} = \int dp \ \bra{x_{1}} e^{- \frac{i}{\hbar} \frac{\hat{p}^{2}}{2m} t'}\ket{p} \bra{p} \ket{x_{0}} = \\
        = \int dp \ e^{- \frac{i}{\hbar} \frac{p^{2}}{2m} t'} \bra{x_{1}}\ket{p} \bra{p} \ket{x_{0}} = \int dp \ e^{- \frac{i}{\hbar} \frac{p^{2}}{2m} t'} e^{\frac{i}{\hbar}p (x_{1} - x_{0})} = \\
        = \int dp \ e^{- \frac{i}{\hbar} \big[\frac{p^{2}}{2m} t' - p (x_{1} - x_{0})\big]} = 
        \begin{vmatrix}
            \big[\frac{p^{2}}{2m} t' - p (x_{1} - x_{0})\big] =  \\
            \frac{t'}{2m} \big[p - \frac{m}{t'}(x_{1} - x_{0})\big]^{2} - \frac{m}{2t'}(x_{1} - x_{0})^{2} \\
        \end{vmatrix} = \\
        = e^{\frac{im}{2\hbar t'} (x_{1} - x_{0})^{2}} \int dp \ e^{- \frac{i t'}{2 \hbar m} \big[p - \frac{m}{t'}(x_{1} - x_{0})\big]^{2}} \\
    \end{gathered}
\end{equation*}

In the integral we can substitute $p \to p + - \frac{m}{t'}(x_{1} - x_{0})$ without the change of bounderies. TODO imaginary gaussian integral

\begin{equation}
    \label{eq:free_particle_propagation_amplitude}
    A_{1} = \sqrt{\frac{2 \hbar m \pi}{i t'}} e^{\frac{im}{2\hbar t'} (x_{1} - x_{0})^{2}}
\end{equation}

From the final result \ref{eq:free_particle_propagation_amplitude}  we see that the probability of meassuring the
transition $(x_{0}, 0) \to (x_{1}, t')$ will be

\begin{equation*}
    P_{1} \sim \frac{1}{t'}
\end{equation*}

thus for an arbitrary choice of $(x_{1} - x_{0})$ there is no fixed time $t'$. This results in a possibility for the particle
being outside of the lightcone with a non-zero probability. This result clearly violates the causality. 

TODO: the transition with the relativistic hamiltonian
