\chapter{Failures of the Relativistic Quantum Mechanics}

In this chapter we will discuss the problems of relativistic quantum mechanics. We will start
with the problem of probability density. We will show that the probability density is positive-definite for Dirac Equation
but is not in the case of the Klein-Gordon equation. Then we will TODO:

\section{Probability density}

\paragraph{} In the non-relativistic case we can find the continuity equation by taking the complex conjugate
of the Schrodinger's equation.

\begin{equation*}
    \begin{gathered}
        i \hbar \partialder{}{t} \psi = \hat{H} \psi \\
        - i \hbar \partialder{}{t} \psi^{*} = \big(\hat{H} \psi \big)^{*}
    \end{gathered}
\end{equation*}

Then we mutiply the first equation by $\psi$ and the second one by $\psi^{*}$.

\begin{equation*}
    \begin{gathered}
        i \hbar \psi^{*} \partialder{}{t} \psi = \psi^{*} \hat{H} \psi \\
        i \hbar \psi \partialder{}{t} \psi^{*} = \psi \big(\hat{H} \psi \big)^{*}
    \end{gathered}
\end{equation*}

If we substract these two equations from each other and use the formula for derivative $(a \cdot b)' = a' \cdot b + a \cdot b'$
we wil get a form from which it will be easy to identify the continuity equation $\partialder{\rho}{t} + div \vec{j} = 0$.

\begin{equation*}
    i \hbar \partialder{\left|\psi\right|^{2}}{t}  = \psi^{*} \hat{H} \psi - \psi \big(\hat{H} \psi \big)^{*}
\end{equation*}

We assume the potential $V$ in $\hat{H} = \frac{\hat{p}^{2}}{2m} + V$ to be real-valued thus the right-hand side
will simplify to

\begin{equation*}
    \psi^{*} \frac{\hbar^{2}}{2m} \nabla^{2} \psi - \psi \frac{\hbar^{2}}{2m} \nabla^{2} \psi^{*} 
    = \frac{\hbar^{2}}{2m} \bigg( \psi^{*} \nabla^{2} \psi - \psi \nabla^{2} \psi^{*} \bigg)
\end{equation*}

and the desired form is

\begin{equation*}
    \partialder{\left|\psi\right|^{2}}{t} + div \bigg[ \frac{\hbar}{2mi} \bigg( \psi \nabla \psi^{*} - \psi^{*} \nabla \psi \bigg) \bigg] = 0
\end{equation*}

Now we will do the same for the Klein-Gordon equation \ref{eq:klein_gordon}. Let's do the same procedure here and
find the complex conjugate, multiply both equations by $\psi$ and $\psi^{*}$ and then substract them. After that 
let's focus on the term which contains the time derivative a thus will represent the \textit{probability density}.

\begin{equation*}
    \partialder{\rho}{t} \sim \frac{1}{2i} \bigg[ \psi^{*} \partialdernd{}{t} \psi - \psi \partialdernd{}{t} \psi^{*} \bigg]
\end{equation*}

Now we can find what $\rho$ is proportial to. 

\begin{equation}
    \rho \sim \frac{1}{2i} \bigg[ \psi^{*} \partialder{}{t} \psi - \psi \partialder{}{t} \psi^{*} \bigg]
\end{equation}

Let's divide the wave function $\psi$ into the real and imaginary part.

\begin{equation*}
    \begin{gathered}
        \psi = i \psi_{1} + \psi_{2} \\
        \rho \sim \bigg[ \psi_{2} \partialder{}{t} \psi_{1} - \psi_{1} \partialder{}{t} \psi_{2} \bigg]
    \end{gathered}
\end{equation*}

Now it is easy to see the problem. In the non-relativistic case the probability density was simply $|\psi|^{2}$ which is
non-negative for any $\psi: \mathbb{R} \to \mathbb{C}$. In the case of Klein-Gordon equation, we can see the probability 
density is zero whenever the wave function is either exclusively real or imaginary. Moreover, since both 
$\partialder{\psi_{1}}{t}$ and $\partialder{\psi_{2}}{t}$ are arbitrary, the value can be negative. This is in contradiction
with the intepretation of $\rho$ as the \textit{probability density of measuring the system within a given state $\psi$}.
To solve this problem, Dirac introduced his equation. Let's check the probability density using the procedure of deriving
the continuity equation from the Dirac equation \cite{dirac_equation_history}. Let's start with the Dirac equation of the 
form as follows.

\begin{equation}
    \label{eq:dirac_equation}
    (i \hbar \gamma^{\mu} \partial_{\mu} - mc) \psi = 0
\end{equation}

For further discussion let's introduce some useful theorems.

\begin{theorem}
    \label{th:gamma_matrices}
    Let $\gamma^{\mu}$'s be Dirac matrices \cite{gamma_matrices} defined by the properties (see \cite{dirac_matrices})

    \begin{equation}
        \label{eq:gamma_matrices_def_1}
        \begin{gathered}
            \big\{\gamma^{\mu}, \gamma^{\nu}\big\} = 2 \eta^{\mu \nu} I_{4}
        \end{gathered}
    \end{equation}

    where $\eta^{\mu \nu} = diag\{+1, -1, -1, -1\}$, and

    \begin{equation}
        \label{eq:gamma_matrices_def_2}
        \begin{gathered}
            \big(\gamma^{\mu}\big)^{\dagger} = \gamma^{0} \gamma^{\mu} \gamma^{0}
        \end{gathered}
    \end{equation}

    then following equations hold for $\mu \in \{0, 1, 2, 3\}$, $i \in \{1, 2, 3\}$.

    \begin{equation}
        \label{eq:gamma_matrices_1}
        \gamma^{0} \gamma^{i} = - \gamma^{i} \gamma^{0}
    \end{equation}

    \begin{equation}
        \label{eq:gamma_matrices_2}
        \begin{gathered}
            \big(\gamma^{0}\big)^{\dagger} = \gamma^{0} \\
            \big(\gamma^{i}\big)^{\dagger} = - \gamma^{i}
        \end{gathered}
    \end{equation}

\end{theorem}

\begin{proof}
    The proof of \ref{eq:gamma_matrices_1} is a direct application of the definition \ref{eq:gamma_matrices_def_1}.

    \begin{equation*}
        \begin{gathered}
            \{\gamma^{0}, \gamma^{i}\} = \gamma^{0} \gamma^{i} + \gamma^{0} \gamma^{i} = 2 \eta^{0 i} I_{4} \\
            \gamma^{0} \gamma^{i} + \gamma^{0} \gamma^{i} = 0 \\
            \gamma^{0} \gamma^{i} = - \gamma^{0} \gamma^{i} 
        \end{gathered}
    \end{equation*}

    To proof \ref{eq:gamma_matrices_2} we just need to know that $\gamma^{0} \gamma^{0} = I_{4}$ which can be shown to be true because we
    constructed the matrices to be so or by checking the definition \ref{eq:gamma_matrices_def_1}: 
    $\{\gamma^{0}, \gamma^{0}\} = 2 \gamma^{0} \gamma^{0} = 2 \eta^{00} I_{4}$. Then we use the defining property 
    \ref{eq:gamma_matrices_def_2} and the equation \ref{eq:gamma_matrices_1}.

    \begin{equation*}
        \begin{gathered}
            \big(\gamma^{0}\big)^{\dagger} = \gamma^{0} \gamma^{0} \gamma^{0} = \gamma^{0} I_{4} = \gamma^{0} \\
            \big(\gamma^{i}\big)^{\dagger} = \gamma^{0} \gamma^{i} \gamma^{0} = - \gamma^{0} \gamma^{0} \gamma^{i} = - I_{4} \gamma^{i} = - \gamma^{i}
        \end{gathered}
    \end{equation*}

\end{proof}

We will use a similar trick we used within the Schrodinger equation or the Klein-Gordon equation. We
take the Hermitian adjoint of the left-hand side of the Dirac equation \ref{eq:dirac_equation}.

\begin{equation*}
    \begin{gathered}
        \big[i \hbar \gamma^{\mu} \partial_{\mu} \psi - mc \psi\big]^{\dagger} = \\
        = \big[i \hbar \gamma^{0} \partial_{0} \psi + \gamma^{1} \partial_{1} \psi + \gamma^{2} \partial_{2} \psi + \gamma^{3} \partial_{3} \psi - mc \psi\big]^{\dagger} = \\
        = - i \hbar \big[\big(\gamma^{0} \partial_{0} \psi\big)^{\dagger} + \big(\gamma^{1} \partial_{1} \psi\big)^{\dagger} + 
            \big(\gamma^{2} \partial_{2} \psi\big)^{\dagger} + \big(\gamma^{3} \partial_{3} \psi \big)^{\dagger}\big] - mc \psi^{\dagger} = \\
        = - i \hbar \big[ \partial_{0} \psi^{\dagger} (\gamma^{0})^{\dagger} + \partial_{1} \psi^{\dagger} (\gamma^{1})^{\dagger} + \partial_{2} \psi^{\dagger} (\gamma^{2})^{\dagger}
            + \partial_{3} \psi^{\dagger} (\gamma^{3})^{\dagger} \big] - mc \psi^{\dagger}
    \end{gathered}
\end{equation*}

By using the equation \ref{eq:gamma_matrices_2} we evaluate all the gamma matrices with a dagger.

\begin{equation}
    \label{eq:conjugate_dirac_equation}
    - i \hbar \big[ \partial_{0} \psi^{\dagger} \gamma^{0} + \partial_{1} \psi^{\dagger} (- \gamma^{1}) + \partial_{2} \psi^{\dagger} (-\gamma^{2})
        + \partial_{3} \psi^{\dagger} (-\gamma^{3}) \big] - mc \psi^{\dagger} = 0
\end{equation}

Now, the trick is to introduce $\overline{\psi} := \psi^{\dagger} \gamma^{0}$, multiply the equation \ref{eq:conjugate_dirac_equation} by 
$\gamma^{0}$ from the right and then rewrite all the $\psi^{\dagger} \gamma^{0}$ terms with $\overline{\psi}$ using the equation \ref{eq:gamma_matrices_1}.

\begin{equation*}
    \begin{gathered}
        - i \hbar \big[ \partial_{0} \psi^{\dagger} \gamma^{0} \gamma^{0} + \partial_{1} \psi^{\dagger} (- \gamma^{1} \gamma^{0}) + \partial_{2} \psi^{\dagger} (-\gamma^{2} \gamma^{0})
            + \partial_{3} \psi^{\dagger} (-\gamma^{3} \gamma^{0}) \big] - mc \psi^{\dagger} \gamma^{0} = 0 \\
        - i \hbar \big[ \partial_{0} \psi^{\dagger} \gamma^{0} \gamma^{0} + \partial_{1} \psi^{\dagger} (\gamma^{0} \gamma^{1}) + \partial_{2} \psi^{\dagger} (\gamma^{0} \gamma^{2})
            + \partial_{3} \psi^{\dagger} (\gamma^{0} \gamma^{2}) \big] - mc \psi^{\dagger} \gamma^{0} = 0 \\
        - i \hbar \big[ \partial_{0} \overline{\psi} \gamma^{0} + \partial_{1} \overline{\psi} \gamma^{1} + \partial_{2} \overline{\psi} \gamma^{2}
            + \partial_{3} \overline{\psi} \gamma^{2} \big] - mc \overline{\psi} = 0 \\
    \end{gathered}
\end{equation*}

And finally, we will multiply this equation by $\psi$ from the left.

\begin{equation}
    i \hbar \partial_{\mu} \overline{\psi} \gamma^{\mu} \psi + mc \overline{\psi} \psi = 0
\end{equation}

To get rid of the $mc \overline{\psi} \psi$ term, let's take the original equation \ref{eq:dirac_equation} and multiply it
by $\overline{\psi}$ from the right. Then we will simply add these two equations

\begin{equation*}
    \begin{gathered}
        i \hbar \partial_{\mu} \overline{\psi} \gamma^{\mu} \psi + mc \overline{\psi} \psi = 0 \, \land \,
        i \hbar \overline{\psi} \gamma^{\mu} \partial_{\mu} \psi - mc \overline{\psi} \psi = 0 \implies \\
        \partial_{\mu} \overline{\psi} \gamma^{\mu} \psi + \overline{\psi} \gamma^{\mu} \partial_{\mu} \psi = 0
    \end{gathered}
\end{equation*}

\begin{equation}
    \label{eq:dirac_continuity}
    \partial_{\mu} (\overline{\psi} \gamma^{\mu} \psi) = 0
\end{equation}

From the continuity equation \ref{eq:dirac_continuity} we see that the probability density should be $\rho = \overline{\psi} \gamma^{0} \psi$.
Which is

\begin{equation*}
    \rho = \overline{\psi} \gamma^{0} \psi = \psi^{\dagger} \gamma^{0} \gamma^{0} \psi = \psi^{\dagger} \psi
\end{equation*}

and thus always a positive quantity! TODO: discussion on the results.

\section{Ground state}

TODO: show negative energies for dirac and KG

\section{Causality}

Now we would like to investigate the problem of causality in the quantum mechanics. We're going to study the transition
amplitude of a free particle to propagate from one space-time point to another one. We will assume the standard definition
of the causility in physics. TODO: discussion on the causality definition.

\begin{definition}
    \label{df:causality}
    Suppose we understand what is meant by a \textit{cause} $C$ and an \textit{effect} $E$. If $C$ happens before $E$ and 
    $E$ is in the future lightcone from the perspective of $C$ then there might be a causality between $C$ and $E$.  Or
    formulated more usefuly for the computation - if $X_{C} = (0, 0)$ is the $C$'s space-time event and 
    $X_{E} = (x_{E}, ct_{E})$ is the $E$'s space-time event then $X_{C}$ can cause the $X_{E}$ if $t_{E} > |x_{E}|$.
\end{definition}

We will follow the same calculation like in \cite{peskin_schroeder}. In the correspondence with the fact that the space translation
is generated by the momentum operator $\hat{p}$, the time evaluation is generated by the energy operator $\hat{H}$. So
if we know the state of the system $\ket{x_{0}} = \ket{x(t=0)}$, the state in the future will be 
$\ket{x_{0}} = \ket{x(t=t')} = \hat{U}(0, t') \ket{x_{1}} = \exp{- \frac{i}{\hbar} \hat{H} t'} \ket{x_{0}}$. The amplitude of the
system going from the state $\ket{x_{0}}$ to $\ket{x_{1}}$ is 

\begin{equation}
    \bra{x_{1}} \exp{- \frac{i}{\hbar} \hat{H} t'}\ket{x_{0}}
\end{equation}

which we will evaluate using the identity $\hat{I}$ expressed in the continuous basis $\{\ket{p}\}$ and the expression for
$\bra{p}\ket{x} = \exp{- \frac{i}{\hbar} p x}$. We will assume a single particle without an external potential thus the
Hamiltonian is simply $\hat{H} = \frac{\hat{p}^{2}}{2m}$.

\begin{equation*}
    \begin{gathered}
        A_{1} = \bra{x_{1}} e^{- \frac{i}{\hbar} \frac{\hat{p}^{2}}{2m} t'}\ket{x_{0}} = \int_{\mathbb{R}} dp \ \bra{x_{1}} e^{- \frac{i}{\hbar} \frac{\hat{p}^{2}}{2m} t'}\ket{p} \bra{p} \ket{x_{0}} = \\
        = \int_{\mathbb{R}} \frac{dp}{2 \pi \hbar} \ e^{- \frac{i}{\hbar} \frac{p^{2}}{2m} t'} \bra{x_{1}}\ket{p} \bra{p} \ket{x_{0}} = \int_{\mathbb{R}} dp \ e^{- \frac{i}{\hbar} \frac{p^{2}}{2m} t'} e^{\frac{i}{\hbar}p (x_{1} - x_{0})} = \\
        = \int_{\mathbb{R}} \frac{dp}{2 \pi \hbar} \ e^{- \frac{i}{\hbar} \big[\frac{p^{2}}{2m} t' - p (x_{1} - x_{0})\big]} = 
        \begin{vmatrix}
            \big[\frac{p^{2}}{2m} t' - p (x_{1} - x_{0})\big] =  \\
            \frac{t'}{2m} \big[p - \frac{m}{t'}(x_{1} - x_{0})\big]^{2} - \frac{m}{2t'}(x_{1} - x_{0})^{2} \\
        \end{vmatrix} = \\
        = e^{\frac{im}{2\hbar t'} (x_{1} - x_{0})^{2}} \int_{\mathbb{R}} \frac{dp}{2 \pi \hbar} \ e^{- \frac{i t'}{2 \hbar m} \big[p - \frac{m}{t'}(x_{1} - x_{0})\big]^{2}} \\
    \end{gathered}
\end{equation*}

In the integral we can substitute $p \to p + - \frac{m}{t'}(x_{1} - x_{0})$ so that the boudaries remain unchanged. Now
the problem is that with the pure imaginary exponent we have an osciallatory function with no other term to guarantee
the integral converges. As was shown in \cite{imaginary_gaussian_integral}, we can actually use the trick of \textit{rotating}
the $p$ in the complex plane by $\frac{\pi}{4}$ and get a known form of a gaussian integral multiplied by the phase factor.

\begin{equation*}
    \begin{gathered}
        = \begin{vmatrix}
            p = e^{\frac{- i \pi}{4}} r
            dp = e^{\frac{- i \pi}{4}} dr
        \end{vmatrix}
        = e^{\frac{i \pi}{4}} e^{\frac{im}{2\hbar t'} (x_{1} - x_{0})^{2}} \int_{\mathbb{R}} \frac{dp}{2 \pi \hbar} \ e^{- \frac{t'}{2 \hbar m} r^{2}} \\
    \end{gathered}
\end{equation*}

\begin{equation}
    \label{eq:free_particle_propagation_amplitude}
    A_{1} = \sqrt{\frac{\hbar m}{2 \pi i t'}} e^{\frac{im}{2\hbar t'} (x_{1} - x_{0})^{2}}
\end{equation}

From the final result \ref{eq:free_particle_propagation_amplitude} we see that the probability density of meassuring the
transition $(x_{0}, 0) \to (x_{1}, t')$ will be

\begin{equation*}
    P_{1} \sim \frac{1}{t'}
\end{equation*}

thus for an arbitrary choice of $(x_{1} - x_{0})$ there is no fixed time $t'$. This results in a possibility for the particle
being outside of the lightcone with a non-zero probability. This result clearly violates the causality. But the conslusion is
not unexpected in the case of the non-relativistic free particle. We used the definition within the relativity theory to 
formulate the condition for the causality but the Hamiltonian $\hat{H}$ doesn't event include the $c$ constant. We could try
to get a proper result by assuming the relativistic energy \ref{eq:relativistic_energy}. As we have shown, a quantization of such an
expression lead us to the Klein-Gordon equation which has the Lorentz symmetry a thus is a good candidate for the right transition
amplitude.

\begin{equation}
    \label{eq:relativistic_energy}
    E^{2} = \vec{p}^{2} c^{2} + m^{2} c^{4}
\end{equation}

The hamiltonian for such a case is $\hat{H} = c\sqrt{\hat{p}^{2} + (mc)^{2}}$. The new amplitude $A_{2}$ is then given by

\begin{equation*}
    \begin{gathered}
        A_{2} = \bra{x_{1}} e^{- \frac{i}{\hbar} c\sqrt{\hat{p}^{2} + (mc)^{2}} t'}\ket{x_{0}} = \int_{\mathbb{R}} dp \ \bra{x_{1}} e^{- \frac{i}{\hbar} c \sqrt{p^{2} + (mc)^{2}} t'} \ket{p} \bra{p} \ket{x_{0}} = \\
        = \int_{\mathbb{R}} \frac{dp}{2 \pi \hbar} \ e^{- \frac{i}{\hbar} c \sqrt{p^{2} + (mc)^{2}} t'} e^{\frac{i}{\hbar}p (x_{1} - x_{0})}
    \end{gathered}
\end{equation*}

Let's analyze this integral in two ways. Firstly, let's try to solve the problem numerically to get the idea of how the integral behaves.
The problem is similiar we had to deal with within the previous integral. The integrated function is oscillatory and thus the convergence
is not expected in general. Numerically, we'll try to deal with that problem by choosing suitable boudaries. The best choice seems to be
$p \in (-mc, mc)$ since it setups the desired condition on the \textit{velocity} to be smaller then the speed of light. Then we only need 
to split the integral into a real and imaginary part.

\begin{equation*}
    \begin{gathered}
        \int_{\mathbb{R}} \frac{dp}{2 \pi \hbar} \ e^{- \frac{i}{\hbar} c \sqrt{p^{2} + (mc)^{2}} t'} e^{\frac{i}{\hbar}p (x_{1} - x_{0})} = \\
        = \begin{vmatrix}
            k_{t'}(p) = - \frac{1}{\hbar} c \sqrt{p^{2} + (mc)^{2}} t' \\
            l_{x_{1}}(p) = \frac{1}{\hbar}p (x_{1} - x_{0}) \\
        \end{vmatrix} \to \\
        \to \int_{-mc}^{mc} \frac{dp}{2 \pi \hbar} \ \bigg[ \sin(k) \sin(l) - \sin(k) \sin(l) + i\big(\sin(k) \cos(l) + \cos(k) \sin(l)\big) \bigg]
    \end{gathered}
\end{equation*}

Now, we can integrate the function numerically using the Python\footnote{The script uses \textit{scipy}, \textit{matplotlib} and \textit{numpy} packages. 
The used interpretter was \textit{cpython}, version 2.7} program \ref{code:free_relativistic_particle_amplitude} because we integrate the 
real part and the imaginary part separatelly. The graph contains the final probability amplitude.

\begin{figure}[h]
    \centering
    \includegraphics[width=0.6\textwidth]{free_relativistic_particle.png}
    \caption{Propagation amplitude for a free relativistic particle}
    \label{fig:free_relativistic_particle_probability}
\end{figure}

\clearpage

\begin{code}
    \captionof{listing}{Numerical integration of the relativistic free particle propagation amplitude}
    \label{code:free_relativistic_particle_amplitude}
    \begin{minted}{Python}
from numpy import sqrt, cos, sin, linspace, exp, pi
import scipy.integrate as integrate
import matplotlib.pyplot as plt

hbar, c, m = 1, 1, 1
TIME = 100

def function(x, t):
    def real(p):
        k = - 1 / hbar * c * (sqrt(p ** 2 + (m * c)**2)) * t
        l = 1 / hbar * p * x
        return 1 / (2 * pi * hbar) * (
            cos(k) * cos(l) - sin(k) * sin(l)
        )

    def imaginary(p):
        k = - 1 / hbar * c * (sqrt(p ** 2 + (m * c)**2)) * t
        l = 1 / hbar * p * x
        return 1 / (2 * pi * hbar) * (
            sin(k) * cos(l) + cos(k) * sin(l)
        )

    return real, imaginary

x_list = linspace(0, 3 * TIME, 500)
y_list = []

for x in x_list:
    real, imaginary = function(x, TIME)

    r = integrate.quad(real, -c * m, c * m)
    i = integrate.quad(imaginary, -c * m, c * m)
    y_list.append(sqrt(r[0] ** 2 + i[0] ** 2) / 2)

plt.xlabel("x-coordinate for $t = {}c$".format(TIME))
plt.ylabel("propagation probability")
plt.plot(x_list, y_list)
plt.plot([TIME, TIME], [0, max(y_list) * 1.2])
plt.gca().set_ylim([0, max(y_list) * 1.2])
plt.show()
    \end{minted}
\end{code}

\clearpage

In the graph \ref{fig:free_relativistic_particle_probability} we also plotted the vertical line which marks the point from which 
the event is not considered causal by our definition and the probability density should be zero there. Althought the value
in the graph is reaching the zero very quickly it is very small but non-zero in the region outside of the lightcone (i.e. on the 
right side of the vertical marking line). So we encouter the same problem as in the calculation with the non-relativistic
particle. The problem of the numericall approach is we have to deal with the non-zero error of the computation. To make
sure the integral is non-zero in the region outside of the lightcone we will investigate the integrall analytically using the 
\textit{steepest descent method}. Let $S(p)$ be the exponent part of the integrated function and $\Delta x = x_{1} - x_{0}$. During
the computation we will assume $(\Delta x)^{2} > (ct')^{2}$ to take into account that we investigate the region where the amplitude
should be zero.

\begin{equation*}
    S(p) = - \frac{i}{\hbar} c \sqrt{p^{2} + (mc)^{2}} t' + \frac{i}{\hbar}p \Delta x
\end{equation*}

Now we should find the saddle points of $S$.

\begin{equation*}
    \begin{gathered}
        \partialder{S(p)}{p} = 0 \\
        p^{2} = \frac{(m c \Delta x)^{2}}{(c t')^{2} - (\Delta x)^{2}}, \ p_{\pm} = \pm \frac{i m c \Delta x}{\sqrt{(\Delta x)^{2} - (c t')^{2}}}
    \end{gathered}
\end{equation*}

Now we will investigate whether the values $p_{+}$, $p_{-}$ are saddle points by finding the Hessian $\partialdernd{S(p)}{p}$.

\begin{equation*}
    \begin{gathered}
        \partialdernd{S(p)}{p} = i \frac{\big((\Delta x)^{2} - (c t')^{2}\big)^{\frac{3}{2}}}{m (ct')^{2}}
    \end{gathered}
\end{equation*}

TODO: why to pick the $+$ solution? Thus we can conclude the amplitude $A_{2}$ is 

\begin{equation}
    \label{eq:a_2_amplitude}
    A_{2} \sim e^{S(p_{+})} = e^{- \frac{mc}{\hbar} \sqrt{(\Delta x)^{2} - (ct')^{2}}}
\end{equation}

The expression \ref{eq:a_2_amplitude} is non-zero for an arbitrary choice of $\Delta x$ and $t'$ thus it is
non-zero also in the region outside of the lightcone. Thus we see we deal with the same problem again.
The difference is that now we deal with an unexplainable problem. We used the relativistic energy and still the theory
allowed the particle to travel faster then the speed of light with non-zero probability! 
