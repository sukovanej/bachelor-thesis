\chapter*{Introduction}
\addcontentsline{toc}{chapter}{Introduction}

\paragraph{} The Quantum Mechanics (QM) is a fundamental theory in physics which aims to describe
the world at the lowest scales of measured units. It had a great success of explaining
a lot of new observations in the very beginning of 20th century, e.g. spectra of atoms,
the black-body radiation or photoelectric effect. In 1905, Albert Einstein published his
famous article \textit{On the Electrodynamics of Moving Bodies} and proposed the Special 
theory of relativity (STR). STR corrects the classical mechanics in a way that it can predict
a state of a physical system having \textit{relativistic velocities} and setups a 
new restrictions for mathematical formulations of physical theories.

A logical step is to develop a physical theory which preserves STR and is capable of 
describing particles in quantum mechanics. This theory is known as Relativistic Quantum 
Mechanics (RQM) and it has been successful in prediction of interesting phenomena in physics
like antimatter or spin. Nevertheless, RQM can't be introduced without formal inconsistencies
and it cant deal with varying number of particles. The modern framework which consistently 
combines QM and STR is the Quantum Field Theory (QFT).

\paragraph{} The thesis consists of three main chapters. In the first one we will discuss the formulation
of RQM, i.e. we will construct two relativistic quantum mechanical eqautions - the Klein-Gordon 
equation and the Dirac equation. In the second chapter we will focus on the failures of these 
formulations and we show the need for a new theory. In the last chapter we will briefly introduce
the new framework of QFT. In the terms of Quantum scalar fields we will try to check whether
the new formulation solves the problems described in the second chapter.
